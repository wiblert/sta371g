\documentclass{beamer}\usepackage[]{graphicx}\usepackage[]{color}
%% maxwidth is the original width if it is less than linewidth
%% otherwise use linewidth (to make sure the graphics do not exceed the margin)
\makeatletter
\def\maxwidth{ %
  \ifdim\Gin@nat@width>\linewidth
    \linewidth
  \else
    \Gin@nat@width
  \fi
}
\makeatother

\definecolor{fgcolor}{rgb}{0.345, 0.345, 0.345}
\newcommand{\hlnum}[1]{\textcolor[rgb]{0.686,0.059,0.569}{#1}}%
\newcommand{\hlstr}[1]{\textcolor[rgb]{0.192,0.494,0.8}{#1}}%
\newcommand{\hlcom}[1]{\textcolor[rgb]{0.678,0.584,0.686}{\textit{#1}}}%
\newcommand{\hlopt}[1]{\textcolor[rgb]{0,0,0}{#1}}%
\newcommand{\hlstd}[1]{\textcolor[rgb]{0.345,0.345,0.345}{#1}}%
\newcommand{\hlkwa}[1]{\textcolor[rgb]{0.161,0.373,0.58}{\textbf{#1}}}%
\newcommand{\hlkwb}[1]{\textcolor[rgb]{0.69,0.353,0.396}{#1}}%
\newcommand{\hlkwc}[1]{\textcolor[rgb]{0.333,0.667,0.333}{#1}}%
\newcommand{\hlkwd}[1]{\textcolor[rgb]{0.737,0.353,0.396}{\textbf{#1}}}%
\let\hlipl\hlkwb

\usepackage{framed}
\makeatletter
\newenvironment{kframe}{%
 \def\at@end@of@kframe{}%
 \ifinner\ifhmode%
  \def\at@end@of@kframe{\end{minipage}}%
  \begin{minipage}{\columnwidth}%
 \fi\fi%
 \def\FrameCommand##1{\hskip\@totalleftmargin \hskip-\fboxsep
 \colorbox{shadecolor}{##1}\hskip-\fboxsep
     % There is no \\@totalrightmargin, so:
     \hskip-\linewidth \hskip-\@totalleftmargin \hskip\columnwidth}%
 \MakeFramed {\advance\hsize-\width
   \@totalleftmargin\z@ \linewidth\hsize
   \@setminipage}}%
 {\par\unskip\endMakeFramed%
 \at@end@of@kframe}
\makeatother

\definecolor{shadecolor}{rgb}{.97, .97, .97}
\definecolor{messagecolor}{rgb}{0, 0, 0}
\definecolor{warningcolor}{rgb}{1, 0, 1}
\definecolor{errorcolor}{rgb}{1, 0, 0}
\newenvironment{knitrout}{}{} % an empty environment to be redefined in TeX

\usepackage{alltt}
\usepackage{../371g-slides}
% Uncomment these lines to print notes pages
% \pgfpagesuselayout{4 on 1}[letterpaper,border shrink=5mm,landscape]
% \setbeameroption{show only notes}
\title{Probability Review 2}
\subtitle{Lecture 2}
\author{STA 371G}
\IfFileExists{upquote.sty}{\usepackage{upquote}}{}
\begin{document}
  
  

  \frame{\maketitle}

  % Show outline at beginning of each section
  \AtBeginSection[]{ 
    \begin{frame}<beamer>
      \tableofcontents[currentsection]
    \end{frame}
  }

  %%%%%%% Slides start here %%%%%%%

   \begin{darkframes}
  

	\begin{frame}[label=lists]{Sample vs Population}
    	Find out the average house price in Austin. 
    	
    	How would you do that? 
    	% Assume that no one has done this before, so it is your duty.
    	
		\begin{figure} 
			\centering
			\scalebox{.4}{ \includegraphics{austin.jpg} }
			\setlength\fboxsep{0pt}
			\setlength\fboxrule{0.5pt}
		\end{figure}  
			
		
		Look at each house price? 
		
		360,000 houses in Austin!  
		
		Can we do something smarter? 
		
     
      
    \end{frame}    
    
    
    
    \begin{frame}[label=lists]{Sample vs Population}
    	A smarter approach:
   		\begin{itemize}
   			\item Pick $n$ houses randomly (e.g. $n=100$)
   			\item Take the average of the prices of these $n$ houses
   			\item Hope that your estimate is close to the true price average.
   		\end{itemize}
   		
   		% Emphasize that the sample should be selected randomly to avoid any biases.
   		
		Just like making polls to predict election results!
		
    \end{frame}    



    \begin{frame}[label=lists]{Sample vs Population}
    	
		
		%\begin{itemize}
   		%	\item Population $\leftarrow$ all houses in Austin
		%	\item Sample $\leftarrow$  $n$ houses you picked
	%		\item Population mean $\leftarrow$ Average house price in Austin
	%		\item Sample mean $\leftarrow$ Your estimate 			
   		%\end{itemize}
   		
		\begin{table}[!b]
        {\carlitoTLF % Use monospaced lining figures
        \begin{tabularx}{\textwidth}{Xcc}
          \textbf{} & \textbf{Population} & \textbf{Sample} \\
          \toprule
          Members       		& all house prices  & prices you picked  \\
          Average               & population mean   & sample mean  \\
          Variance          & population variance   & sample variance \\
          \bottomrule
        \end{tabularx}}
        
      \end{table} 
      \quad \newline    		
   		
   		Estimating a \alert{population parameter} (population mean) based on a \alert{sample statistic} (sample mean).
		
    	%We could also estimate other population parameters, such as variance using the sample variance.
		
    \end{frame}  
 

    \begin{frame}[label=lists]{Collecting a sample}
    
    \begin{columns}[onlytextwidth]
        \column{.45\textwidth}
        	Zillow.com, ``Austin, TX.'' 
        	\begin{itemize}
   				\item Click ``More Map''
   				\item Select 15 houses, note their prices in an R script. 
   				\item Do not discard any price, use the first 15
   				\item Try to represent different regions
			\end{itemize}
        
         
        \column{.5\textwidth} 
        	
        	\begin{figure} 
				\centering
				\scalebox{.35}{ \includegraphics{zillow.jpg} }
				\setlength\fboxsep{0pt}
				\setlength\fboxrule{0.5pt} 
			\end{figure} 	
        	        
        \end{columns}
					
    \end{frame}  
    
    
    
    
    	\defverbatim[colored]\sampleZillow{
      \begin{lstlisting}[language=R,tabsize=2]
# Create a vector of house prices (You should have 15 price data)
sample_house_prices <- c(327000,276000,513000)
# Calculate sample statistics
sample_mean <- mean(sample_house_prices)
sample_variance <- var(sample_house_prices)
sample_standard_deviation <- sd(sample_house_prices)
# Sample mean of first 5 houses
sample_mean_5 <- mean(sample_house_prices[1:5])
# Print them to console
cat("Sample Mean", sample_mean)
cat("Sample Variance", sample_variance)
cat("Sample Standard Deviation", sample_standard_deviation)
cat("Sample Mean of first 5 houses",sample_mean_5)

    \end{lstlisting}}	
    
    \begin{frame}[label=lists]{Collecting a sample}
    Your R script should look like this
    \sampleZillow
    
    \end{frame}


    \begin{frame}[label=lists]{Sampling Distribution} 
		On Learning Catalytics, enter your results. \newline
		
		And here is what they look like... % Use sample mean with 15 houses.
    \end{frame}



    \begin{frame}[label=lists]{Sampling Distribution}
    	%Everyone found a different sample mean, which one is correct?
    	%None. \newline %But they should be all clustered around the true mean (the average house price in Austin). \newline
    	
		Distribution of your answers  $\rightarrow$ \alert{Sampling distribution} \newline 
		    	
    	%Sample mean (your answers) itself has a distribution, separate from the house price distribution in Austin. This is called  \alert{sampling distribution}. \newline
    
    	
		\begin{table}[!b]
        {\carlitoTLF % Use monospaced lining figures
        \begin{tabularx}{\textwidth}{Xcc}
          \textbf{Statistic} & \textbf{Population} & \textbf{Sample Mean} \\
          \toprule
          Mean       		& $\mu$  & $\mu$  \\
          %Variance          & $\sigma^2$     & $\sigma^2/n$  \\
          Standard Deviation          & $\sigma$     & $\sigma/\sqrt{n}$ \\
          \bottomrule
        \end{tabularx}}
        
      \end{table}    	
      \quad \newline
      What would you expect if you had 10\,000 houses in your survey?
    	
    	
		
    \end{frame}
    
       
    \begin{frame}[label=lists]{Sampling Distribution}
		Let's compare sample mean of 5 houses vs 15 houses. \newline
		
		What do you expect to see?
	\end{frame}
	
	
	\begin{frame}[label=lists]{Sampling Distribution}
		\begin{figure} 
				\centering
				\scalebox{.5}{ \includegraphics{sampling_1.jpg} }
				\setlength\fboxsep{0pt}
				\setlength\fboxrule{0.5pt} 
			\end{figure} 	
	\end{frame}	
	
	
		\begin{frame}[label=lists]{Sampling Distribution}
		\begin{figure} 
				\centering
				\scalebox{.5}{ \includegraphics{sampling_2.jpg} }
				\setlength\fboxsep{0pt}
				\setlength\fboxrule{0.5pt} 
			\end{figure} 	
	\end{frame}	
    
    
    
    \begin{frame}[label=lists]{Sampling Distribution}
    
    Assume $\mu=$\$300K, $\sigma=$\$60K. \newline
    
   	\begin{table}[!b]
        {\carlitoTLF % Use monospaced lining figures
        \begin{tabularx}{\textwidth}{Xccc}
          \textbf{} &  $n$ & $\sigma/\sqrt{n}$  & 3 std. dev. range (99.7\%) \\
          \toprule
          Survey 1     & 25   &  \$12K	&\$274K ............. \$336K   \\
          Survey 2     & 100  &  \$6K	&\$282K ....... \$318K   \\
		  Survey 3     & 3600 &  \$1K	&\$297K ... \$303K   \\
          \bottomrule
        \end{tabularx}}	
	\end{table}   
    
    \end{frame}
    
    
	
	

    
    
    \begin{frame}[label=lists]{$t$ Distribution}
    
    	We often do not know population variance and use sample variance instead. \newline
    	
    	
		In that case, the sample mean will have a \alert{$t$ distribution}.
	\end{frame}	
    
    
    
    \begin{frame}[label=lists]{Hypothesis Testing}
    Hypothesis: The average house price in Austin is \$1M.
    
    Your survey on 25 houses: Average price is \$$305K$. \newline
    
    Questions, questions...
    \begin{itemize}
   \item Would you reject the hypothesis? Why?
    
   \item Is it possible that, out of bad luck, you picked the cheapest houses?
    
   \item Would you be more comfortable with your conclusion if you had 1000 houses in your survey?
    
   \item When should you reject the hypothesis? When not?
    
   \end{itemize}


	\end{frame}
	
	
	\begin{frame}[label=lists]{P-Value}
		Your sample mean: \$305K. 	\newline
	
		$H_0:$ $\mu=\$1M$	(Null hypothesis)
		
		$H_1:$ $\mu<\$1M$	(Alternative hypothesis) \newline
		
		
		
		The \alert{$P$-value} is ``the probability of observing such an extreme (\$305K or less) sample statistic given the null hypothesis is true.'' \newline
		
		\begin{itemize}
		\item $P$-value  $\leq \alpha$, reject the null hypothesis
		\item $P$-value  $> \alpha$, reject the null hypothesis
		
		\end{itemize}		
		
		
		$\alpha$ is usually chosen as $0.05$ \underline{prior to sampling}.		
		
		% Ask how n would affect p value
		% Emphasize that the interpretation of p value is always the same.

	\end{frame}
	
	
	\begin{frame}[label=lists]{P-Value}
		If the null hypothesis were true...
		\begin{figure} 
				\centering
				\scalebox{.4}{ \includegraphics{pval2.jpg} }
				\setlength\fboxsep{0pt}
				\setlength\fboxrule{0.2pt} 
		\end{figure} 			
		P-value is smaller than $10^{-100}$, while $\alpha=0.05$. 
		
		Rather than thinking you are cursed, you simply reject the hypothesis!
	\end{frame}
	
	

	

	
	\begin{frame}[label=lists]{Confidence Interval}
		Content Here
	\end{frame}
	
	
    	\defverbatim[colored]\confInt{
      \begin{lstlisting}[language=R,tabsize=2]
# Calculate 95% confidence interval (default)
avg_price_ci_95 <- t.test(sample_house_prices)
# Calculate 99% confidence interval
avg_price_ci_99 <- t.test(sample_house_prices, conf.level = 0.99)
# Display results
cat("95% confidence interval is:", avg_price_ci_95$conf.int)
cat("99% confidence interval is:", avg_price_ci_99$conf.int)

    \end{lstlisting}}	
    
    \begin{frame}[label=lists]{Confidence Interval}
    Add the following to your R script
    \confInt
    
    Enter your results on Learning Catalytics.
    
    \end{frame}
	

\end{darkframes}
  
  
  

\end{document}

